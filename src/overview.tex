\section{Overview}\label{overview}

LAOS, the Living Assets Open System,
aims at providing the universal platform where dynamic NFTs are created and evolved for all blockchains
in a truly non-custodial way, in contrast to highly centralized practices common today.
It will introduce and leverage the powerful concept of bridgeless minting and evolution, 
whereby developers can benefit from the features of LAOS while using any blockchain of their choice
to do trading and use DeFi services as usual, with no bridges required.

LAOS will enable DApp developers to build applications that unlock the full power of 
digital ownership, migrating from the common mindset of extreme {\it scarcity} and associated {\it speculative dynamics}, 
to one of {\it abundance} and captured {\it User Generated Value} (UGV).

LAOS will be a fully open layer 1 blockchain, decentralized via usage of its own native utility token,
built as a specialized Parachain in the Polkadot
ecosystem, inheriting its security and advanced features, and with an architecture
designed to scale as usage grows. Within Polkadot, it will leverage its trustless connection
to other Parachains specialized in smart contracts, such as Moonbeam and Astar,
and data storage, such as Crust Network.

LAOS is committed to raising public awareness about the potential risks of centralized flows
and partnering with regulatory bodies to ensure that blockchain-based
digital ownership lives up to its promise; assets that are tradable
for real money and held in privately-owned servers entail
risks akin to those typically associated with securities.

Section \ref{introduction} provides an introduction to the current landscape of NFT technology,
focusing on some of the main issues that have led to limited use cases, and highly centralized
practices. 
Section \ref{living-assets} introduces the main aspects and enabled use cases around
Living Assets.
%
Section \ref{sec:architecture} provides an in-depth walkthrough of the core 
technology and architecture of LAOS, as well as the main actors involved and their incentives.
Section \ref{sec:bridgeless-tech} describes the patterns that allow LAOS to implement
bridgeless minting and evolution. 
%
Section \ref{security} presents an analysis of attack vectors and their
potential mitigation. 
%
Section \ref{conclusion} concludes with a summary
of the main aspects presented.