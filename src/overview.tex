\section{Overview}\label{overview}

LAOS aims to become the consensus system that every other chain uses to scale their Digital Ownership transactions,
instantly and securely, without bridges, and in a truly non-custodial way, in contrast to highly centralized practices common today.

It introduces and leverages the powerful concept of bridgeless minting and evolution, 
whereby developers can benefit from the features of LAOS, from day 1, while remaining in the blockchain of their choice
to do trading and use DeFi services as usual, with no bridges required.

LAOS enables DApp developers to build applications that unlock the full power of 
digital ownership, migrating from the common mindset of extreme {\it scarcity} and associated {\it speculative dynamics}, 
to one of {\it abundance} and captured {\it User Generated Value} (UGV).

LAOS is a fully open layer 1 blockchain, decentralized via usage of its own native utility token,
built as a specialized Parachain in the Polkadot
ecosystem, inheriting its security and advanced features. LAOS will replicate a part of Polkadot's architecture
inside Polkadot itself, providing its own sharding-based scaling specialized in Digital Ownership. 

Within Polkadot, it leverages its trustless connection
to other Parachains specialized in smart contracts, such as Moonbeam and Astar,
and data storage, such as Crust Network.

LAOS provides the foundations for companies and creators to
protect themselves from legal issues regarding their assets being considered securities,
by storing asset metadata on decentralized systems that can be verified on-chain.
Committed to promoting awareness of the risks of centralized flows, LAOS collaborates with
regulators to ensure blockchain digital ownership lives up to its promise,
ensuring companies do not become custodians of the NFTs they create.

Section \ref{introduction} provides an introduction to the current landscape of NFT technology,
focusing on some of the main issues that have led to limited use cases, and highly centralized
practices. 
Section \ref{laos-assets} introduces the key aspects and
use cases enabled by Digital Ownership within LAOS.
%
Section \ref{sec:architecture} provides an in-depth walkthrough of the core 
technology and architecture of LAOS, as well as the main actors involved and their incentives.
Section \ref{sec:bridgeless-tech} describes the patterns that allow LAOS to implement
bridgeless minting and evolution. 
%
Section \ref{security} presents an analysis of attack vectors and their
potential mitigation. 
%
Section \ref{conclusion} concludes with a summary
of the main aspects presented.