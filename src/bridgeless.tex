\section{Bridgeless Minting \& Evolution}\label{sec:bridgeless-tech}

The architecture described in section \ref{sec:architecture} supports 
a pattern capable of enabling the minting and evolution
of assets in LAOS, while their trading (and any other ownership-related
concept) is managed in any other blockchain of choice, as 
long as it supports smart contracts; this applies to both EVM and
non-EVM compatible cases. We stress that this does not require any
type of bridge, neither trusted nor trustless.

We first introduce the concept of
Universal Location in Section \ref{sec:universal-location},
and then describe a simplified version of the pattern which 
only allows bridgeless evolution (Section \ref{sec:bridgless-evo}).
We conclude discussing the full-fledged flow for bridgeless minting and evolution
in Section \ref{sec:bridgless-all}.

\subsection{Universal Location}\label{sec:universal-location}

One recent standard that LAOS will leverage is that of {\it Universal
Location}. It was introduced by Polkadot as a natural evolution
of the {\it MultiLocation} standard that was already in use within their
ecosystem, as part of their Cross-Consensus Messaging (XCM)
pattern\cite{multilocation}.

The fact that the Relay Chain serves as consensus parent for all
Parachains has enabled the design of trustless message-passing patterns
among Parachains, for example, by relying on the light clients
that collators of Parachains have from other Parachains.

MultiLocation assists this communication by identifying
every single location that exists within the world of consensus
systems that share a common parent, and example being all Parachains,
as their finality derives from the consensus in the parent Relay Chain.
A location in a consensus system is defined as an
{\it isolatable state machine} held within global consensus.

For instance, the following:

\begin{align}
    & \text{MultiLocation =}  \label{eq:multilocation1} \\
    & \,\, ../\text{Parachain}(42)/\text{AccountKey20}(0x12...cd) \nonumber
\end{align}

can be used by one Parachain to refer to a 20-byte account ($0x12...cd$)
that exists within a different Parachain (42), that shares the same parent consensus
system: the Relay Chain (at '$../$').

By contrast, Polkadot's XCMv3\cite{multilocationXCMv3} introduced 
the concept of a {\it Universal Location} under which all systems
which generate their own consensus exist, and by extension,
all possible locations within consensus. The Universal Location is the only location
that has no parent. In typical filesystem formatting, the Universal Location 
can be thought of as the {\it root},
often represented by the starting '$/$' symbol of a path.

Universal Location enables the creation of routes that
connect smart contracts in an origin blockchain
to assets or accounts in, e.g., Polkadot, through references such as:
\begin{align}
     & \text{MultiLocation =}  \label{eq:multilocation2} \\
     & \,\, /\text{Polkadot}/\text{Parachain}(42)/\text{AccountKey20}(0x12...cd) \nonumber
\end{align}

Finally, the {\it MultiAsset} part of the standard uses these concepts to
standardize the referral to both fungible and non-fungible tokens.

\subsection{Bridgeless Evolution}\label{sec:bridgless-evo}

The simplest usage of LAOS for developers that want the
ownership of assets to be managed by a different blockchain, e.g., Ethereum,
is to simply mint assets in the origin blockchain, and 
use a LAOS MultiLocation string, using the Universal Location '$/$' as root,
to specify the tokenURI, 
effectively using \bref{eq:multilocation2} to create a concrete implementation
of pattern \bref{eq:tokenURIConsensus}:
\bea
\mbox{tokenURI:} \mbox{ tokenId} &\rightarrow & \mbox{LAOS MultiLocation}
\nn 
\label{eq:tokenURILaos}
\eea 
As Figure \ref{fig:patterns2nodesAs1} depicts, all that DApps need 
to build applications that use this pattern 
is to query a LAOS node to resolve tokenURIs and provide asset metadata.

This pattern basically links the asset metadata to an updatable slot governed
by a consensus system, with transparency, traceability,
data availability for both current 
and past states. And it does so in a flow which, ultimately, does not require of 
any trusted party.

In this pattern, however, asset minting is still required in the origin
blockchain, potentially running into scalability issues or incurring 
in high gas prices. 

\subsection{Bridgeless Minting \& Evolution}\label{sec:bridgless-all}

The previous pattern becomes more powerful when {\it minting} of assets,
as opposed to only evolution, can also be offloaded to LAOS by any 
other blockchain. We shall first discuss the nomenclature and 
the building blocks required to build this functionality, and then
analyze the resulting flow.

This process requires two simple components: a pattern to build the {\it id} 
of assets, and a piece of smart contract logic that utilizes the meaning
that can be extracted from such {\it ids}.  

\subsubsection{Encoding \& Decoding Methods}
The setup starts with the usage of a concrete pattern to
generate {\it ids} for assets, enabling the encoding of a
generic web3 address ({\it w3a}). Let us call $Enc$ and $Dec$ the encoding
and decoding functions, correspondingly:
\bea
Enc: & w3a, \Omega & \rightarrow id \,,
\nn
Dec: & id  & \rightarrow w3a\,,
\label{eq:enc-dec}
\eea
where $\Omega$ represents any other information that the {\it id} may
contain, e.g., a consecutive index often used to distinguish assets as they
are minted. 

$Enc$ is an injective function; in particular, two assets with different {\it id} must 
necessarily differ in either $w3a$ or $\Omega$. Note, however, that there may be
a large number of assets that share the same $w3a$, as made manifest
by the $Dec$ function, which projects out all information conveyed by $\Omega$.

A simple example for an $Enc$ function is provided by the binary concatenation of 
the binary representations of $w3a$ and $\Omega$, which corresponds to a sibling
$Dec$ function that trims the bits assigned to $w3a$.

\subsubsection{Smart Contract Logic}
The ERC721/1155 smart contract logic required in the origin blockchain must 
contain a minor code extension in the implementation of the 
$ownerOf$ (and related) methods. 

For the sake of simplicity, let us assume that the smart contract implements 
a simple method, $localStorage.wasEverTraded(id)$, that returns {\it true} if the asset 
has been traded {\it at least once}. On deploy of the smart contract,
this method naturally returns $false$ for every $id$.

As customary, the contract writes to storage as assets are traded,
allowing a method we shall call {\it localStorage.ownerOf} to return
the new owners, and reflecting in positive returns of $wasEverTraded$.
These methods can be used to extend $ownerOf$ as follows:

\begin{algorithm}[H]
    \SetKwFunction{GetOwner}{ownerOf(id)}
    \SetKwData{wasEverTraded}{localStorage.wasEverTraded(id)}
    \SetKwData{localStorage}{localStorage.ownerOf(id)}
    \SetKwData{defaultOwner}{Dec(id)}
    
    \SetKwProg{Procedure}{method}{}{}
    \Procedure{\GetOwner}{
        \If{\wasEverTraded}
        {
            \KwRet \localStorage\;
        }
        \Else{
            \KwRet \defaultOwner\;
        }
    }
\end{algorithm}

\subsubsection{Resulting Flow}

The final ingredient required for bridgeless minting is implemented in LAOS,
by simply using exactly the same encoder/decoder functions \bref{eq:enc-dec} to name assets
in the Evochains, and declaring assets in these particular collections as {\it foreign 
assets}, signalling that they cannot be traded in the LAOS Ownership Chain.

The aforementioned smart contract logic, together with the LAOS MultiLocation
pattern \bref{eq:tokenURILaos}, allows one single deploy of an ERC721/1155
contract in an origin blockchain to link all assets to LAOS, while retaining 
the management of all ownership-related aspects in the origin chain. 
In effect, all assets ever assignable to all potential initial owners are reserved on deploy,
and the LAOS consensus system is assigned the responsibility to {\it fill-in} the
corresponding initially-empty registers as assets are minted, and modify them
as they are evolved.

Again, figure \ref{fig:patterns2nodesAs1} illustrates a convenient way to architecture
nodes that would serve bridgeless minting and evolution; in this case, the ownership
node would be one of the origin chain.
 