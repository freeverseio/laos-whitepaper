\section{Security}\label{security}

LAOS core components are based on software and consensus protocols
that are already in use,
and analyzed extensively by other teams.
In this sense, LAOS security analysis is different from 
projects proposing entirely new, still untested, protocols;
in LAOS case, the analysis mostly consists of, for every component
in its architecture, relying on previously published
comprehensive studies of their attack vectors and potential mitigation.

\subsection{Ownership Parachain}\label{sec:security-parachain}

The Ownership Parachain will
implement the default Parachain pattern that relies on the Polkadot
Relay Chain to provide block finality \cite{parachain-protocol}, \cite{parachain-protocol2}.
This protocol enables efficient heterogeneous sharding,
and has been live in mainnet, in Kusama first, since
October'19, and then in Polkadot, since March'20.

Parachains are free to decide their runtime and are encouraged to focus on 
specialization, in contrast to the one-runtime-fits-all
homogeneous sharding being built in Ethereum 2.0.
LAOS will diligently adhere to this pattern,
specializing in all aspects relevant to Digital Ownership.
The Ownership Chain will use the default Parachain pattern:
Nominated Proof of Stake (NPoS) consensus algorithm,
with GRANDPA providing finality over BABE block production.
We refer the reader to the thorough theoretical analysis presented in\cite{polkadotExplained}
for full details.

It is essential to note that despite the freedom that Parachains have to design
their inner details, every single transaction of every Parachain is eventually
verified by the required majority of validators in the Relay Chain before
being included in a Polkadot block. 

Since the consolidation in Polkadot of Parachain's 
invalid transactions is prevented by the Relay Chain, 
the only significant attack vectors relevant to Parachains
are related to censorship and data availability.

Censorship corresponds to Parachain collators not including one or more
transactions based on criteria different from those specified by the
protocol, e.g., based on the sender address. To prevent these attacks,
a Parachain only needs to ensure that there are some neutral collators,
but not necessarily a majority.
Theoretically, the censorship problem is solved with 
having just one honest collator,
as analyzed in\cite{collators}.

Data availability attacks appear if the Parachain collators collude to 
withhold block data, or the corresponding Proof of Validity (PoV), to the 
validators in the Relay Chain. 
Polkadot's protocol mitigates this 
attack severely by using Erasure Coding\cite{al2018fraud}, which
adds redundancy on the block/PoV data 
made availability by a minimal subset of honest collators, and enables Relay Chain
validators to reconstruct its entirety out
of a small set of randomly queried data.

In summary, the standard Parachain-in-Polkadot level of security 
protects all features assigned to the Ownership Parachain,
including transfers of LAOS assets and tokens among owners as well as to
sibling Parachains, management of the LAOS token, 
transaction fees, treasury funding, storage of LAOS Relay Chain and Evochain states,
and management of their runtimes.

\subsection{Trustless Bridge}\label{sec:security-bridge}

The design of the bridge that will connect the Ownership Chain 
and the LAOS Relay Chain is built around the same technology that 
will connect Polkadot and Kusama via BridgeHub\cite{bridgehub}.

This bridge design decouples the consensus mechanism
and the state machine, requiring both networks to run 
{\it on-chain consensus clients}, also known as {\it on-chain light clients},
that only track consensus proofs of state
transitions instead of the full state transitions.
References \cite{SeunLanlege} and \cite{snowfork} provide a good introduction to 
known attack vectors to light client based bridges in systems
that either do not track the signatures of all validators, or do 
not implement accountability on validators that provide multiple signatures
over different block proposals. 

Instead, the bridge implementation is based on the on-chain light
client used by Polkadot and Kusama~\cite{parity-bridges}, 
designed only for systems whose finality is provided by the GRANDPA finality gadget.
Eventually, the bridge will become more efficient when Polkadot integrates
{\it on-chain accountable light clients}, which use signature
verification via Zero-Knowledge schemes, and introduce on-chain
accountability mechanisms. We refer the reader to~\cite{ciobotaru2022accountable}
for a thorough analysis of attack vectors.

Regarding data withholding attack vectors,
it is important to consider that any actor can 
relay block data and the corresponding transition proofs
between the Ownership Chain and the LAOS Relay Chain.
The attack can be straightforwardly mitigated by having at least one
actor with incentives in the LAOS platform running a node of the
LAOS Relay Chain, even if in read-only mode, capable of relaying 
the Evochain data as new blocks are produced.

Moreover, the LAOS Relay Chain validators have their own incentives
to relay this data because they receive rewards
in the form of LAOS tokens
upon inclusion of new valid blocks in the Ownership Chain.


\subsection{LAOS Relay Chain}\label{sec:security-laos-relay}
 
The LAOS Relay Chain is an instance of Polkadot's Substrate-based Relay Chain,
with GRANDPA finality.
As such, its security depends
on the standard game-theoretical incentive system that
leads to more than two-thirds of their validators following
the protocol honestly.

A comprehensive analysis of GRANDPA, including a listing of 
attack vectors and their potential mitigation, can be found in\cite{grandpa}.

LAOS Relay Chain validators will be rewarded for including valid blocks
from the Evochains, through transaction fees in LAOS tokens obtained from
the Ownership Chain via the trustless bridge. Likewise, they will be rewarded
when new block headers of the LAOS Relay Chain are correctly included in the Ownership Chain,
in this case, in native LAOS tokens. 
This incentive system encourages LAOS Relay Chain validators
to not only provide data to the bridge relayers
but also potentially take on the role of bridge relayers themselves.

\subsection{Evochains}\label{sec:security-evochains}

Evochains are Parachains of the LAOS Relay Chain,
supporting asset mint and evolution at protocol level.
As such, the same security considerations discussed for the Ownership Parachains
(Section ~\ref{sec:security-parachain}) apply to the Evochains, replacing
Polkadot's validators by LAOS Relay Chain validators.