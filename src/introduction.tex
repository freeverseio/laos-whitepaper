\section{Current NFT Landscape}\label{introduction}

\subsection{Scalability and Use-Cases}\label{current-nfts}

During 2017, blockchain-based Non-Fungible Tokens (NFT) became popular within the crypto 
community, especially after the initial success of Cryptokitties (CK) \cite{kitties-contract}. 
CK and its successors largely set the mindset of the first wave of 
NFT-based applications; it is therefore important to understand the technical limitations that these
had to put up with.

Despite the fact that, compared to non-blockchain games, CK was an extremely limited game,
with only a handful of actions available to users, which were often executed within days of separation,
it managed to collapse the underlying Ethereum blockchain
\cite{buterin2013whitepaper}, \cite{jiang2021cryptokitties}, \cite{kitties-kraken}
when peaking around
15K daily active addresses, accounting for 25\% of all Ethereum transactions.
Fast forwarding 5 years, another game, Sunflower Farmers, reached a 40\% quota of all
gas consumed by Polygon, a scaling layer-2 on top of Ethereum, with roughly 5x more activity 
\cite{sunflower-play2earn}, \cite{sunflower-coindesk}.

These severe scalability limitations inevitably impacted on the use cases initially
implemented by different industries. On the one hand, {\it extreme scarcity} became the most exploited
pattern, since minting and trading require on-chain usage\cite{chohan2021non}.
While prices of real-world products necessarily derive from their production and
distribution costs, the price of the first wave of blockchain digital items was largely
unrelated to such costs. Artificial digital scarcity was used to increase initial selling
price, and led to purchasers {\it speculating} on 
the possibility of reselling the same item for a greater profit
(see \cite{pinto2022nft} for an interesting mathematical analysis on the expectation
of profits during the first wave of NFTs).

\subsection{Mutability and Centralization}\label{tokenURI}

Another pattern that arose due to lack of scalability was that of considering NFTs as
either {\it static} collectibles or as fully centralized digital goods; indeed, in many cases,
as both. This pattern was initiated with the rapid adoption of the ERC721 standard
\cite{erc721}, which contains an {\it optional} method
that returns a Universal Resource Identifier (URI) for every token:
\be \label{eq:tokenURI}
\mbox{tokenURI:} \mbox{ tokenId} \rightarrow \mbox{string}
\ee 
\noindent The standard suggested the following use: "This allows your smart contract to be interrogated [...] for details about the assets which your NFTs represent".

While other methods in the ERC721 standard, such as {\it ownerOf}, allow the blockchain to be
queried about {\it who owns} an asset, the optional {\it tokenURI} method aims at
being queried about {\it what is owned}. 

The vast majority of DApps ended up choosing one of the following approaches:
\bea
\label{eq:tokenURIIPFS}
\mbox{tokenURI:} \mbox{ tokenId} &\rightarrow & \mbox{IPFS address} 
\\
\label{eq:tokenURICentral}
\mbox{tokenURI:} \mbox{ tokenId} &\rightarrow &\mbox{private URL} 
\eea 

The first choice \bref{eq:tokenURIIPFS} is often made for static, immutable,
NFTs that represent assets of high value; a representative example is Beeple's
{\it Everydays—The First 5,000 Days}\cite{beeple}, sold at Christie’s for \$69.3M \cite{kugler2021non}.
In this case, the answer to {\it what is owned} refers users to the Inter-Planetary
Filesystem (IPFS \cite{benet2014ipfs}), which has the following two essential properties:
\begin{itemize}
    \item it is a {\it content addressed} system, whereby the address of a digital file
    is the result of applying a certain Hash function $H$: {IPFS address = $H$(content)};
    \item it is decentralized in nature, due to its peer-to-peer protocol.
\end{itemize}
When using the IPFS pattern \bref{eq:tokenURIIPFS}, the first property ensures that the blockchain 
returns an address that uniquely describes the content of the NFT. Together with the 
second property, the result is that {\it what is owned} can be answered in a fully 
trustless manner, without the need to query any trusted 3rd party.

The downside of pattern \bref{eq:tokenURIIPFS} is that it leads to a clash between asset 
mutability and scalability. Since any minor change in the asset's
content would necessarily map it to a different IPFS address, an 
on-chain transaction would need to be used on every change of every asset. 
Thus, pattern \bref{eq:tokenURIIPFS} is mostly used for immutable NFTs.

The centralized pattern \bref{eq:tokenURICentral} has been the choice for many DApps,
especially those that required some degree of asset mutability.
It allows developers to minimize the use
of the blockchain, by simply storing a link to external, privately-owned URLs, which
host {\it what is owned}, the asset metadata. A representative example 
is Sorare \cite{sorare},
a popular web3-based sports fantasy game, where the data of the NFTs
is stored in the company's private servers\footnote{As a concrete example, at the time of writing,
the blockchain method {\it tokenURI} for a Lionel Messi NFT \cite{messiOpenSea} pointed to
this Sorare's owned URL \cite{messiOnSorare}}

With the centralized pattern \bref{eq:tokenURICentral}, mutability is simply achieved by changing the data
provided by each company's private servers. This has several drawbacks. The most important
consequence is a critical loss of the core principles
of blockchain technology, as users must rely on private companies
for an understanding of their own holdings, thereby undermining decentralization.
Furthermore, these companies can change such data without leaving any trace, to the extreme 
of making it unavailable. Such events can even happen involuntarily, for example, 
via hacks, or because a company ceases operations, or is forced to. 
Private companies become the custodian of the attributes, and by extension,
the value of the NFT, and assume the risks associated to it. 
We shall refer to this particular highly centralized practice as the {\it elephant in the room}.

\subsection{Insecure Bridges}\label{bridges}

Finally, the aforementioned limitations have also forced DApp developers 
to make undesired compromises when selecting the blockchain to build on.
Concerns around gas costs, availability of the underlying coin, user base 
of the existing DeFi ecosystem, or degree of decentralization and
security are weighted against the needs of each specific application.

Bridges are often built to allow users to migrate assets across blockchains,
which tends to be a complex UX process.
But building bridges is hard. Almost every relevant
bridge built to date requires some level of centralization,
and many of them have been hacked\cite{lee2022sok},
resulting in millions of stolen funds. Examples include
widely used bridges between Ethereum and Solana\cite{eth-solana},
bridges required to play {\it Axie Infinity},
the most popular NFT-based game at the time\cite{axie-bridge-1}, \cite{axie-bridge-2},
as well as vulnerabilities of up to \$850M that 
were live for some time in the Polygon to Ethereum bridge,
but fixed before being exploited\cite{polygon-hack}.


\subsection{Growing Legal Concerns}\label{summary}

As discussed, initial limitations of fully programmable-blockchain technology
led to many of the early NFT use cases relying on some combination of
extreme scarcity and speculation, making choices between an immutable approach, 
or highly centralized practices, and often forcing users to go through insecure bridges.

Legislation is indeed starting to pay closer attention to centralized NFT flows. A relevant
example is the recent class-action lawsuit filed against the popular NFT-based application 
NBA Top Shot\cite{nba}, to decide whether or not their NFTs should be considered
unregistered securities\cite{dapper-labs-nft-ruling}. Although this lawsuit 
does not deal with the {\it elephant in the room} directly, the judge 
did argue that users were forced to rely on the running of the platform by a few
private actors,
leading to {\it expectation of profit being “derived from the efforts of others”}.

Although legislation in the web3 space is a highly complex, rapidly evolving field,
it may well be the case that similar NFT-as-securities lawsuits follow soon, 
given the aforementioned custodian role that many private companies are currently assuming. 



